\documentclass[titlepage]{report}

\usepackage{titlesec}
\usepackage{lipsum}
\usepackage{graphicx}
\usepackage{wrapfig}
\usepackage[export]{adjustbox}
\usepackage[dvipsnames]{xcolor}
\usepackage{colortbl}

%setting for chapter font
\titleformat{\chapter}[display]
  {\Huge\bfseries}{}{1pt}{\Huge}
  
  

\title{\textbf{Milestone 1}}  


\begin{document}
\maketitle

\tableofcontents{}
\newpage

\chapter{Plan for milestone 1}
In this phase, we develop a plugin for OpenVINO and using Tensor Virtual Machine (TVM), we generate kernels in order to run a simple topology on a single FPGA. The generated kernels are customized for OpenVINO and synthesized for Stratix 10.
\chapter{Development of OpenVINO Plugin}
Firstly, the deep learning deployment toolkit (dldt) github repository was integrated into the university gitlab server. A folder named Noctua\_plugin was created inside dldt and another folder named kernels was created inside Nocuta\_plugin to store the bitstreams (.aocx files).\par
The development process started by editing the classification sample application to parse the IR of the given input topology. This parsed IR, which is stored in a data structure called CNNNetwork, is then passed on to the plugin. A single file named fpga\_plugin.cpp retrieved the layer information from the IR, launched the kernels on the FPGA after passing them the required arguments.


\chapter{Using TVM to generate kernels}


\chapter{Summary of the progress}
The developed plugin supports convolution, Max pooling and Fully connected layers as required by the simpleCNN topology.


\chapter{Conclusion and feedback}


\end{document}